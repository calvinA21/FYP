\chapter{Background}%
\label{chp:background}

\section{Related Works}%
\label{sec:related}

In the literature, there is no absence of works in the field of \gls{ml}-based
network intrusion detection. This section, will briefly outline some
significant works in the field.

In 2020, Karatas et al.~\cite{Karatas} proposed 6 \gls{ml}-based \gls{ids},
employing the techniques of K Nearest Neighbours, Random Forest, Gradient
Boosting, Adaboost, Decision Tree and Linear Discriminant Analysis. They
evaluated their models on the CSE-CIC-IDS 2018 dataset. Due to the class
imbalance present in this dataset, the authors applied the \gls{smote}. The
results indicate the proposed models are very effective with accuracies on the
sampled data ranging from 91.18\% to 99.35\%. The Random Forest algorithm had
the highest accuracy on the sampled dataset, however the Adaboost algorithm had
the highest precision and F1 score which may be more significant given the
class imbalance present in the dataset.

Jiang et al. (2020)~\cite{Jiang} take a \gls{dl} approach to the problem. They
propose a \gls{cnn} to extract spatial features, which are then processed
through a \gls{bilstm} network to extract temporal features. To handle class
imbalance, the authors first apply \gls{oss} to reduce noisy samples in the
majority class, then increase the number of minority samples through
\gls{smote}. The proposed solution is evaluated on both the NSL-KDD dataset and
the UNSW-NB15 dataset, achieving accuracies of 83.58\% and 77.16\%
respectively.

Mighan and Kahani (2021)~\cite{Mighan} propose a hybrid approach using a
stacked auto-encoder network for latent feature extraction followed by a
support vector machine classifier. The proposed model was evaluated on the
ISCX2012 and CICIDS2017 datasets achieving accuracies of 90.2\% and 99.49\%
respectively. They concluded that \gls{dl} feature extraction outperformed
other methods. Note, class imbalance was not addressed in this study, despite
being present in both datasets used.

Kus et al. (2022)~\cite{Kus} argue that the current evaluation standard in
\gls{ml}-based industrial intrusion detection may create a false sense of
security. They argue that it does not assess the model's ability to detect
unknown attacks, absent from the training set. They propose a new evaluation
methodology to assess the efficacy of \gls{ml}-based industrial \gls{ids} in
detecting these unknown attacks.

The methodology involves training and testing the model on variants of the
dataset. The procedure consists of 4 phases. In the first phase, one attack
category is excluded from the training set each iteration. Hence, the
generalisability to this unseen attack can be measured from the evaluation on
the test set. In the second phase, only one attack category is included in the
training set, allowing the relations between different attack categories to be
investigated. Finally, in the third and fourth phase, the first two phases are
repeated however using individual attacks in place of attack categories. This
results in a number of dataset variants equal to 4 times the number of attack
categories.

This methodology is applied to three industrial intrusion detection models
proposed in a work by Lopez Perez et al.~\cite{Perez} The results indicate an
alarmingly low ability to recognise unseen attacks, with detection rates
dropping to between 3.2\% and 14.7\% for some types of attacks. The authors
conclude that the models they tested only learn signatures of attacks and are
not performing true anomaly-based intrusion detection. The study focuses
primarily on industrial intrusion detection and does not attempt to make any
generalisations to other domains of intrusion detection.

A literature review carried out in 2023 by Ahmad et al.~\cite{zero-day} reviews
many more similar works with a particular focus on zero-day attacks. It
supports the argument of Kus et al.~\cite{Kus} stating that many researchers in
the field are not currently addressing the existence of zero-day attacks and
suggests that future works should not assume all existing attack
classifications are present in the dataset.

Pu et al.~\cite{} propose an unsupervised approach, combining \gls{ssc} and
\gls{ocsvm}. They evaluate their proposal on the NSL-KDD dataset against three
other unsupervised models, specifically, a combination of \gls{ssc} and
\gls{ea}, \gls{dbscan} and K-means clustering. The results indicated the
proposed model outperformed the other models.

Cao et al.~\cite{} take an unsupervised deep learning approach, proposing new
regularisers to classical and variational \gls{ae}s to map normal network
samples towards the origin. One class classification models can then be used to
detect malicious samples which will be mapped further from the origin. They
evaluated this approach of 4 datasets, namely, CTU13\-09, CTU13\-13, NSL-KDD,
and UNSW-NB15. The results indicate the approach is effective on high
dimensional and sparse network data.

In 2023, Zoppi et al.~\cite{Zoppi} carried out a comparison of 47 different
machine learning algorithms on 11 different network intrusion datasets. The
study includes supervised, unsupervised, deep learning and meta-learning
algorithms. The methodology, like that of Kus et al.\ involves creating
variants of each dataset excluding specific categories in order to simulate the
occurrence of unknown attacks. They conclude that supervised classifiers yield
a higher accuracy, however, this is significantly degraded in the face of
unknown attacks. Unsupervised learners on the other hand are less accurate but
experience a less dramatic decline in the face of unknown attacks. They also
conclude meta-learners outperform their non-meta-learning counterparts.

\section{Conclusion}%
\label{sec:conclusion}
From the above works, we observe that the research community has achieved
very impressive results in the field of intrusion detection. However, as Kus et
al.\ have discovered, these laboratory results may not accurately reflect
reality.

The supervised learning approaches explored above do not consider the
possibility of unknown attacks and hence may be ineffective in practical
applications. As indicated by Zoppi et al.~\cite{Zoppi}, unsupervised
techniques may be more effective in detecting unknown attacks, however at the
cost of reduced efficacy against known attacks.
