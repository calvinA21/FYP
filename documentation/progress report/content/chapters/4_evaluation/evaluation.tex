\chapter{Proposed Evaluation}%
\label{chp:evaluation}

The work of Kus et al.\ shines an illuminating light onto the field, however is
limited to the scope of industrial intrusion detection and the single work on
which their methodology was evaluated. Hence, this study will apply the same
methodology, however with a wider scope.

\section{Models}%
\label{sec:models}

The methodology will be applied to several state of the art models designed for
more general purpose datasets. Namely, the study will reproduce the models
proposed in the works of Karatas et al.~\cite{Karatas} and Jiang et
al.~\cite{Jiang} which were discussed in the background section. These works
have been selected as they are the most highly cited amongst recent studies
that provide clear methodologies and were evaluated on realistic datasets. In
addition, both these studies address class imbalance, which is a significant
problem when applying \gls{ml} to intrusion detection due to the prevalence of
normal, non-malicious traffic in realsitic scenarios when compared to the
volume of malicious traffic~\cite{imbalance}.

\section{Datasets}%
\label{sec:datasets}

The UNSW-NB15 dataset was generated on the IXIA PerfectStorm tool. The Tcpdump
tool was used to capture 100GB of raw traffic. This traffic was processed into
4 csv files containing a collective 2,540,044 records with 49 features. The
dataset contains diverse attack scenarios including nine attack categories.
Namely, these are fuzzers, analysis (e.g.\ portscan), backdoors, \gls{dos},
exploits, generic (e.g.\ brute force), reconaissance, shellcode and worms. The
simulation consisted of three virtual networks, two of which generated normal
traffic and one which generated the malicious traffic. The attack behaviour is
derived from the CVE site in order to ensure the attack scenarios are
realistic~\cite{unsw15}.

The CSE-CIC-IDS2018 dataset is a collaborative effort between the
Communications Security Authority and the Candian Institute of Cybersecurity,
and serves as a successor to the CICIDS2017 dataset which is a smaller, less
diverse counterpart. The aim of the project was to produce an up to date,
realistic dataset for the evaluation of \gls{nids}. The dataset was generated
on infrastructure consisting of a victim network with 420 machines and 30
servers, and an attack network consisting of 50 machines. A typical network
environment was simulated using B-profiles, which describe user behaviour
through statistical and \gls{ml} techniques, allowing it to be accurately
reproduced. The dataset includes a variety of modern attack categories, namely,
\gls{dos} attacks, brute force attacks, code injection, botnet attacks, and
infiltration attacks which involve using an exploit on a host inside the
network to attack the network. The resulting dataset contains 80 features and
approximately 16,000,000 instances~\cite{cic2018, cic2018data}.

Both these datasets provide a large volume of data containing diverse attack
scenarios and a realistic network environment. It is for these reasons that
they will be used to evaluate the models that will be reproduced from the work
of Karatas et al.\ and Jiang et al. Furthermore, the CSE-CIC-IDS2018 dataset
is, to the best of my knowledge, the most up-to-date, general purpose network
intrusion dataset publicly available today. The UNSW-NB15 dataset is also one
of the more recent publicly available general purpose datasets. Note, each
selected model includes one of these datasets in the original authors'
evaluation. This allows for comparison of results to verify that the models
were accurately reproduced.

\section{Methodology}%
\label{sec:methodology}

Once the models have been reproduced, the procedure proposed by Kus et al.\
will be carried out. In the ideal case, the methodology will be applied to both
models on both datasets, necessitating four complete runs of the procedure and
the evaluation metrics of each iteration would be calculated using 5-fold cross
validation. However, due to the high computational demands of the methodology,
there is a possibility that the proposed procedure will be unfeasible to carry
out using the resources available at the University of Malta. Should this be
the case, the UNSW-NB15 dataset can be omitted. The model proposed by Jiang et
al.\ will be validated on the UNSW-NB15 to verify accurate reproduction and all
following experiments will be carried out solely on the CSE-CIC-IDS2018
dataset. If further reduction in computation time is required, a standard
train-test split may be employed in place of k-fold cross validation and
stratified sampling can be explored to reduce the size of the dataset without
decreasing the quality of the results.

\section{Metrics}%
\label{sec:metrics}

The performance of each iteration of training will be evaluated using standard
metrics used in classification tasks in the field of \gls{ml}. These include
accuracy, precision, recall and F1 score values. As well as the
receiver-operating characteristic curve and the area under this curve.

The recall calculated can then be used to generate a heatmap for each phase of
the methodology proposed by Kus et al. The resulting heatmaps will clearly
illustrate which attacks and categories were detected even when absent from the
training set and which attacks and categories contain information about other
attacks.

\section{Drawing Conclusions}%
\label{sec:conclusions}

The first research question defined in the aims and objectives section can be
addressed by an aggregate of all the results. If overall, models achieve high
detection rates on attacks omitted from the training set, we can conclude that
current state of the art models can indeed detect never-seen-before attacks. If
models on the otherhand achieve low metrics, we can conclude that they cannot
detect never-seen-before attacks.

To address the second research question, we can compare the performance of the
models proposed by Karatas et al.\ with that of the model proposed by Jiang et
al. Karatas et al.\ take a traditional \gls{ml} approach whereas Jiang et al.\
take a \gls{dl} approach, hence, comparision of these results will address the
second research question.

This study will investigate the relations between groups of attacks which will
be referred to under a descriptive name that we will call a category. As Ahmad
et al.~\cite{zero-day} noted, categorising attacks into broad labels such as
\gls{dos} can create inconsistencies and confusion in the literature due to
their subjectivity. Hence, it should be noted that this categorisation should
be interpreted as a shorthand method of referring to the specific attack types
included under the cateogry in this study. Conclusions will not be intended to
generalise to attack types not present in the dataset that may fall under the
same subjective categorisation.
