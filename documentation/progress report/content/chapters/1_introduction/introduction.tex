\chapter{Introduction}%
\label{chp:introduction}

The ever-evolving landscape of cyber threats poses a pervasive danger to all
facets of society, ranging from commercial enterprises to individuals and
government entities. As technology relentlessly advances and more aspects of
our lives become digitised, the risks posed by these threats continue to
escalate~\cite{ENISA}. There are numerous defence options available to help
mitigate these threats, prominent among them are \gls{nids}.

\gls{nids} are a crucial part of any information system's defence that aims to
identify and mitigate network threats in real time. Two major
classifications exist that characterise the function of intrusion detection
systems.\ \gls{sids} detect attacks by matching sequences of code or commands
with those of known attacks.\ \gls{aids} on the other hand, detect threats by
recognising deviations from normal, non-malicious traffic~\cite{survey1}.

Designing \gls{nids} suitable for the ever-evolving cyberthreat landscape is no
trivial task.\ \gls{nids} deployed practically are likely to face attacks not
known to the developer. These will be referred to as unknown or unseen attacks.
These could take the form of new deviations of known attacks, new attacks based
on known vulnerabilities or zero-day attacks, which are cyberattacks executed
on the basis of an unknown exploit. Due to their nature, \gls{sids} are
ineffective in detecting unknown attacks, as by definition, they cannot exist
on any attack signature database~\cite{survey1}.  % TODO p : find citation for unknown attacks

\gls{aids} exhibits promise through its ability to identify unknown
attacks~\cite{aids-unseen}, particularly with the advent of \gls{ml}
techniques. The majority of works in the field investigate
the use of supervised learning, which has achieved impressive accuracy.
However, many researchers present their work as a closed-set classification
problem~\cite{zero-day}. A closed-set classification problem is one where the
dataset includes all possible classes the model needs to predict~\cite{closed-set}. This does not
account for the existence of unknown attacks which may present misleading
results as the cyberthreat landscape is ever-evolving.

Numerous approaches have been proposed to detect unknown attacks, however these
will typically come at the cost of a reduced accuracy when considering known
attacks. Hence, the detection of unknown attacks remains an open problem in
intrusion detection.
