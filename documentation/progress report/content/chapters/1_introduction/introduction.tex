\chapter{Introduction}%
\label{chp:introduction}

The ever-evolving landscape of cyber threats poses a pervasive danger to all
facets of society, ranging from commercial enterprises to individuals and
government entities. As technology relentlessly advances and more aspects of
our lives become digitised, the risks posed by these threats continue to
escalate. There are numerous defence options available to help mitigate these
threats, prominent among them are \gls{ids}.

\gls{ids} are a crucial part of
any information system's defence that aims to identify and mitigate threats in
real time. There are two primary domains at which intrusion detection systems
can operate, namely \gls{nids}, which analyse network packets to identify
attacks, and \gls{hids}, which analyse logs and other information sources from
a single host. In addition, two major classifications exist that characterise
the function of intrusion detection systems.\ \gls{sids} detect attacks by
matching sequences of code or commands with those of known attacks.\ \gls{aids}
on the other hand, detect threats by recognising deviations from normal,
non-malicious traffic.

Extensive research exists in the field of \gls{ids}, however, despite these
advancements, successful attacks continue to occur. Zero-day attacks are an
especially challenging problem in the field of cybersecurity which contribute
heavily to these undetected attacks. Zero-day attacks are cyber attacks
executed on the basis of a vulnerability or attack vector that was not
previously known, in essence, they are the first exploitation of a previously
unknown vulnerability. Due to their nature, \gls{sids} are completely
ineffective in detecting zero-day attacks, as by definition, they cannot exist
on any attack signature database.\ \gls{aids} exhibits promise through its
ability to identify zero-day attacks, particularly with the advent of \gls{ml}
techniques, notably \gls{dl}. Extensive research exists in this field in
the hope of detecting and mitigating zero-day attacks, however, despite
numerous implicit assumptions that current \gls{ml}-based \gls{aids}
can detect unseen attacks, little empirical evidence exists supporting this
assumption.
