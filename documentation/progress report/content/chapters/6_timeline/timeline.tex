\chapter{Timeline Going Forward}%
\label{chp:timeline}

Going forward, the next step will be to replicate the results obtained by Kus
et al.~\cite{Kus} from the developed python script. This will verify that the
script implements their methodology correctly.

Once this has been done, the work of Karatas et al.~\cite{Karatas} will be
replicated. This will involve the following steps:
\begin{enumerate}
      \item Applying \gls{smote} on the CSE-CIC-IDS2018 dataset to reduce the class
            imbalance present in the dataset.
      \item Preprocessing the dataset,
            \begin{enumerate}
                  \item Adding an ID column to keep track of samples during preprocessing
                  \item Replacing NaN and infinity values with numerical representations
                  \item Dividing the timestamp column into 2 numerical columns
                  \item Adding sign columns for the `Init Fwd Win Byts' and `Init Bwd Win Byts' columns
                  \item Converting attack labels to numerical values
                  \item Shuffling the dataset
            \end{enumerate}
      \item Implementation of the 6 \gls{ml} models using the scikit-learn library in
            python
      \item Training the models and verifying the results match the original work
\end{enumerate}

Next, the NSL-KDD dataset will be downloaded and labels will be added to it
uniquely identifying attacks. The work of Pu et al.~\cite{Pu} and Cao et
al.~\cite{Cao} can then be replicated.

Replicating the work of Pu et al.~\cite{Pu} will involve the following steps:
\begin{enumerate}
      \item Preprocessing the dataset
            \begin{enumerate}
                  \item Encoding non-numeric categorical features using one-hot encoding
                  \item Applying the F-test for feature selection
                  \item Normalising all features
            \end{enumerate}
      \item Implementation of the \gls{ssc}-\gls{ocsvm} model
      \begin{enumerate}
            \item Dividing the feature set into N subspaces
            \item Training \gls{ocsvm} models for each subspace and producing partitions
            \item Updating a dissimilarity vector with the dissimilarity values from each partition
            \item Ranking the dissimilarity vector and applying a threshold dissimilarity value to flag anomalous samples
      \end{enumerate}
      \item Training the model and verifying the results match the original work
\end{enumerate}

Cao et al.~\cite{Cao} provide a GitHub repository including the
source code required to replicate the work, hence, this repository will be
cloned and used to replicate the model.

Finally, the developed python script can be executed on each model. The
results of these experiments will then be processed into tables and the
heatmaps will be generated from the recall values computed.
