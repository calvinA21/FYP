\chapter{Timeline Going Forward}%
\label{chp:timeline}

Going forward, the next step will be to reproduce the results obtained by Kus
et al.\ from the developed python script. This will verify that the script
implementes their methodology correctly.

Once this has been done, the work of Karatas et al.\ will be reproduced. This
will involve the following steps:
\begin{enumerate}
    \item Applying \gls{smote} on the CSE-CIC-IDS2018 dataset to reduce the class
          imbalance present in the dataset.
    \item Preprocessing the dataset,
          \begin{enumerate}
              \item adding an ID column to keep track of samples during preprocessing
              \item replacing NaN and inifinity values with numerical representations
              \item dividing the timestamp column into 2 numerical columns
              \item adding sign columns for the \'Init Fwd Win Byts' and \'Init Bwd Win Byts'
                    columns
              \item converting attack labels to numerical values
              \item shuffling the dataset
          \end{enumerate}
    \item implementation of the 6 \gls{ml} models using the scikit-learn library in
          python
    \item training the models and verifying the results match the original work
    \item repeating the above steps on the UNSW-NB15 dataset (if computational resources
          allow) in order to compare with the \gls{dl} model.
\end{enumerate}

Next, the UNSW-NB15 dataset will be downloaded and labels will be added to it
uniquely identifying attacks. The work of Jiang et al.\ can then be reproduced,
which will involve the following steps:
\begin{enumerate}
    \item Reducing the imbalance of the UNSW-NB15 dataset using \gls{oss} on the majority
          samples and \gls{smote} on the minority samples.
    \item Preprocessing the data:
          \begin{enumerate}
              \item Converting cateogrical features into one-hot encoding
              \item Normalisation of all features
          \end{enumerate}
    \item Developing the \gls{cnn} using the Tensorflow and Keras libraries in python
    \item Developing the \gls{bilstm} network with the same libraries
    \item Training the model and verifying the results wioth the original work
    \item Repeating the above steps on the CSE-CIC-IDS2018 dataset in order to compare
          with the previous models.
\end{enumerate}

Finally, the developed python script can be executed being given each
combination of dataset and model as parameters. The results of these
experiments will then be processed into tables and the heatmaps will be
generated from the recall values computed.
