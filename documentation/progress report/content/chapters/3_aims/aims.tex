\chapter{Aims and Objectives}%
\label{chp:aims}
\hypertarget{obj}{}

The principal aim of this study will be to investigate the efficacy of various
techniques proposed in the literature against both known and unknown attacks.
This will take the form of two objectives:

\begin{center}
    \fbox{\parbox{0.8\linewidth}{\textbf{Objective 1:} To determine the extent to which current state-of-the-art
            network intrusion detection models are able to detect both attacks that were present and absent in their
            training set.}}
\end{center}

This information is vital to measure the pragmatic performance of these models
as \gls{nids} deployed in practical circumstances will face a wide variety of
attacks, including unknown attacks.

\begin{center}
    \fbox{\parbox{0.8\linewidth}{\textbf{Objective 2:} To investigate the extent to which unsupervised techniques are
            more or less effective against both known and unknown attacks compared to supervised
            techniques.}}
\end{center}

Unsupervised techniques do not rely on labelled attack data in their training.
Hence, they are theoretically better adapted to unknown attacks, however may be
less accurate on known attacks. This hypothesis is supported by the findings of
Zoppi et al.~\cite{Zoppi} and will be further explored in this study.
