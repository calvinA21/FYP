\chapter{Aims and Objectives}%
\label{chp:aims}

% Key can current state of the art models detect unseen attacks are deep learning
% techniques more effective than traditional techniques in detecting unseen
% attacks

% Sidequest are the datasets exploired equally suitable for training models that
% can detect unseen attacks (red flag)

% It time allows: What can be done to increase the ability of a model to detect
% unseen attacks

From the previous section, we can see that the research community has achieved
very impressive results in the field of intrusion detection. However, as Kus et
al.\ have discovered, these laboratory results may not accurately reflect
reality.

\gls{aids} is said to be able to detect zero-day attacks as it categorises
traffic based on its behaviour as opposed to particular
signatures~\cite{aids-zero-day}. However, current state of the art
\gls{ml}-based \gls{ids} are typically trained on labelled data, meaning the
resultant models are classification models as opposed to true anomaly detection
models. This may bring them closer to \gls{sids} than \gls{aids}, significantly
reducing their efficacy in detecting zero-day attacks.

The principle aim of this study, will be to determine whether or not current
state of the art network intrusion detection models can detect attacks that
were not present in their training set.

This should serve as a more realistic measurement of performance as \gls{nids}
deployed in practical circumstances will face a wide variety of attacks, most
of which will not have been present in its training set.

In addition, the study will investigate whether \gls{dl} is more effective in
detecting unseen attacks than traditional \gls{ml} techniques.\ \gls{dl} is
able to model complex patterns and relationships, requiring far less feature
engineering and preprocessing from the developer. Therefore, we hypothesise
these features may make it better equipped to model the complex and abstract
underlying patterns in malicious behaviour. This study will aim to verify this
hypothesis.
