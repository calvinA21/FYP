\chapter*{Abstract}
\addcontentsline{toc}{chapter}{Abstract}

The cybersecurity landscape is in constant flux, with new attacks continuously
developed by threat actors to circumvent system defences. Consequently, network
intrusion detection systems deployed in practice are likely to encounter cyber
threats not known to the developer, i.e., unknown attacks.\ \gls{ml} exhibits
great potential to detect such attacks leveraging its capacity to model complex
patterns without explicit developer knowledge. Numerous machine learning
techniques have been applied in the literature in an effort to achieve this
goal, however, many applications of supervised techniques are presented as
closed set classification problems neglecting the critical aspect of detecting
unknown attacks.

To address this gap, this study assesses and compares the efficacy of several
machine learning models in detecting both known and unknown attacks while
exploring the underlying relationships between different attack types to
facilitate generalisation. The study considers a range of algorithms from three
families of \gls{ml} techniques, namely, traditional supervised techniques,
traditional unsupervised techniques and deep unsupervised techniques.
Experiments are conducted on the CSE-CIC-IDS2018 dataset, often recognised as
the most realistic and up-to-date network intrusion detection dataset currently
available. This dataset, characterized by its diversity, scale, and relevance
to real-world scenarios, provides a robust foundation for our experiments. This
research helps to fill a crucial void in the existing literature by evaluating
\gls{ml} techniques on their ability to detect unknown cyber threats. It also
contributes to advancing the understanding of how these techniques can
effectively be employed in practical cybersecurity applications.

Our results indicate that supervised models are very effective and generally
perform better than unsupervised models on known attacks, however are
ineffective against unknown attacks. Unsupervised models, in contrast, are only
effective on certain categories of attacks but remain equally effective against
both known and unknown attacks.
