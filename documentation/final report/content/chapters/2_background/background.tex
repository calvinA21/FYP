\chapter{Background}%
\label{chp:background}

\section{Network Intrusion Detection}%
\label{sec:nids}

\gls{nids} play a crucial role in the realm of network defence, complementing
other techniques such as firewalls and encryption. These sophisticated software
systems analyse network traffic to detect malicious behaviour in real
time~\cite{survey2}.\ \gls{nids} can be broadly divided into two categories
distinguished by the method of identification employed~\cite{survey1}.

\gls{sids} function by identifying specific attack signatures. They typically
require access to a signature database, recognising previously documented
attacks by correlating sequences of commands or actions to entries on the
database. This is highly effective on known threats present in the database,
however their efficacy wanes considerably against unknown
attacks~\cite{survey1, survey2}.

\gls{aids}, in contrast, discern attacks by profiling non-malicious behaviour
and recognising deviations. This strategy relies on the assumption that
malicious traffic inherently differs from benign activity. This approach yields
several advantages, primarily, the capability to recognise unknown attacks,
absent from any attack signature database and the lack of a need for such a
database, which can be laborious to maintain and keep up-to-date. Of course,
the approach is not without disadvantages, most notably the high susceptibility
to false alarms as any deviation from the expected profile, malicious or not,
will be flagged~\cite{survey1, survey2}.

Numerous methodologies can be employed in the implementation of \gls{aids},
including statistics-based, knowledge-based and \gls{ml}-based
approaches~\cite{survey1}. Statistics-based \gls{ids} rely on statistical
models and tests to isolate outliers from normal data, which are then flagged
as potential threats. Knowledge-based \gls{ids} leverage expert insights or
domain-specific rules to flag behaviour that may be malicious.\ \gls{ml}-based
\gls{ids} employ \gls{ml} models trained to differentiate normal and malicious
traffic patterns~\cite{survey1}.

Several \gls{nids} implementations exist in practice, for example Snort and
Elastic Security are two popular options. Snort is an open-source \gls{nids}
that employs signature-based detection to identify threats through predefined
rules, while also incorporating some anomaly detection capabilities through
preprocessors. Elastic Security is a software tool built on the Elastic Stack,
a powerful data analytics and search platform. Elastic Security allows for
intrusion detection rules to be defined leveraging data from various sources to
identify malicious behaviour. Detection rules offer significant flexibility
allowing for a variety of techniques to be employed including signature-based
detection, knowledge-based detection and \gls{ml}-based rules.

\section{Machine Learning in Network Intrusion
  Detection}%
\label{sec:ml_nids}

\gls{ml} exhibits incredible potential in the field owing to its capacity to
model complex patterns directly from data without developer
knowledge~\cite{ml}.
This could simplify the process of creating \gls{nids} and improve the ability
of these systems to detect unknown attacks. The realm of \gls{ml} encompasses a
diverse array of methodologies and algorithms, broadly classified into
supervised, unsupervised, and reinforcement learning
paradigms~\cite{ml_taxonomy}.

This study will consider supervised and unsupervised techniques. Supervised
techniques rely on labelled data and generate predictions by learning to map
the feature set to the label based on the training data~\cite{supervised_ml,
    ml_taxonomy}. Unsupervised techniques, on the other hand, operate without the
need for labelled data. They instead learn patterns present in the data by
measuring similarity and dissimilarity between data points. These identified
patterns can subsequently be used to inform predictions~\cite{unsupervised_ml,
    ml_taxonomy}.

These families can be further divided into traditional techniques and \gls{dl}
techniques. Traditional techniques typically leverage feature engineering and
statistical methods to extract meaningful information from datasets enabling
predictions or decisions based on the derived information~\cite{Najafabadi2015,
    useful_ml}. In contrast, \gls{dl} techniques consists of numerous layers of
interconnected neurons which each take the sum of their inputs, multiply by
their weight and apply an activation function to the output. This is then
passed either to the neurons of the next layer or to the final output
layer~\cite{dl}.

Supervised techniques tend to be affected negatively by imbalanced datasets,
where they may exhibit a bias towards predicting the majority class, leading to
less accurate predictions for minority classes~\cite{imbalance_prob,
    survey_cicids}. One approach to address this issue, which has been employed
extensively in the literature, is \gls{smote}~\cite{smote, smote_survey,
    Karatas, Jiang}.\ \gls{smote} generates synthetic minority class samples to
reduce imbalance by interpolating data from other nearby points. These points
are selected using the \gls{knn} algorithm~\cite{knn}. By increasing the number
of minority class samples in the dataset, the effects of class imbalance can be
alleviated~\cite{smote}.

\section{Metrics}%
\label{sec:metrics}

Various evaluation metrics exist in the field of \gls{ml}. In the domain of
classification problems, some of the most prevalent metrics used to evaluate
\gls{ml} models include accuracy, precision, recall and F-measure values. These
are calculated from the counts of \gls{tp}, \gls{fp}, \gls{tn} and \gls{fn}
predictions made during testing~\cite{metrics}.

Accuracy~\cite{metrics} represents the proportion of correct predictions and is
the most common metric used. It can, however, be misleading in datasets with
significant class imbalance as if the model performs well on the majority
class, it will achieve a high accuracy regardless of the performance on the
minority class. In tasks like intrusion detection, where detecting minority
classes is crucial, this skewed accuracy can be misleading. It is defined as:

\[ \text{Accuracy} = \frac{\text{\gls{tp}}+\text{\gls{tn}}}{\text{\gls{tp}}+\text{\gls{tn}}+\text{\gls{fp}}+\text{\gls{fn}}} \]

Recall~\cite{metrics} represents the proportion of samples of a specific class
that were correctly identified by the model. In the context of intrusion
detection, recall holds particular importance as it directly influences the
system's capability to detect attacks effectively. It is defined as:

\[ \text{Recall} = \frac{\text{\gls{tp}}}{\text{\gls{tp}}+\text{\gls{fn}}} \]

Recall-Unk~\cite{Zoppi} represents the recall value when considering only
samples of the unknown class. It is proposed by Zoppi et al.~\cite{Zoppi} to
better define the performance of a model on unknown classes and is considered
in this study for the same reasons. Zoppi et al.~\cite{Zoppi} define \gls{tu}
and \gls{fu} in order to calculate Recall-Unk.\ \gls{tu} represent correctly
identified attack instances that belong to an attack excluded from the
training set.\ \gls{fu} represent attack instances belonging to an excluded attack
that are not identified by the model. Recall-Unk is defined as:

\[ \text{Recall-Unk} = \frac{\text{\gls{tu}}}{\text{\gls{tu}}+\text{\gls{fu}}} \]

Precision~\cite{metrics} represents the likelihood a positive prediction made
by the model is correct, an important metric in intrusion detection as a low
precision indicates a high false alarm rate. It is defined as:

\[ \text{Precision} = \frac{\text{\gls{tp}}}{\text{\gls{tp}}+\text{\gls{fp}}}
\]

The F1-measure~\cite{metrics} serves as a unified metric that considers both
recall and precision. It is less susceptible to the effects of class imbalance
when compared to accuracy. It is defined as:
\[ \text{F1-measure} = 2 \times \frac{\text{Precision} \times
        \text{Recall}}{\text{Precision} + \text{Recall}} \]

The \gls{roc} curve~\cite{metrics} is also a common tool used for evaluating
models, with the \gls{auc-roc} serving as a valuable metric. The curve consists
of the \gls{fpr} rate on the x-axis plotted against the recall on the y-axis,
where \gls{fpr} is defined as:
\[ \text{\gls{fpr}} = \frac{\text{\gls{fp}}}{\text{\gls{fp}} + \text{\gls{tn}}}
\]

% K-fold cross-validation is a robust evaluation strategy widely employed in
% machine learning. The dataset is divided into k partitions, or folds, and the
% model is trained k times each time using a different fold as the test set
% whilst all other partitions constitute the training set. The metrics calculated
% in each fold are then aggregated yielding the final results. This enhances the
% reliability of the evaluation by decreasing variance in the metrics caused by
% stochasticity in the train-test splitting of the dataset~\cite{kfoldcv}.

% Note: if the above is uncommented the section should be renamed, otherwise move the paragraph to a different section

\section{Feature Selection}%
\label{sec:feat_eng}
Feature selection is among the most important components of preprocessing.
The aim of feature selection is to simplify the set of features in a dataset and
increase the efficacy of the model by selecting the most relevant features and
discarding unnecessary features~\cite{feat-sel}. Features may be relevant,
redundant, or irrelevant. Relevant features are those that offer valuable
information about the target variable and are vital to enable the model to
learn patterns present in the dataset. Redundant features refer to features
that also offer valuable information to the model however are superseded by
other features that contain the same or more information. Hence, their
inclusion will increase the model complexity without any significant
improvement to model efficacy. Irrelevant features refer to those that do not
contain any information about the target variable. These become a source of
noise to the model and should be removed to maximise efficacy and reduce excess
model complexity.

Feature selection techniques, similarly to machine learning techniques, can be
categorised based on their label requirements. Supervised feature selection
techniques require labels and are typically employed for classification and
regression tasks, whereas, unsupervised techniques do not require labels and
are typically employed for clustering tasks~\cite{feat-sel}.

\section{Literature Review}%
\label{sec:literature}

There exist several works in the literature focused on the field of
\gls{ml}-based network intrusion detection. This section briefly outlines some
significant works in the field.

In 2020, Karatas et al.~\cite{Karatas} proposed six \gls{ml}-based \gls{ids},
employing the techniques of \gls{knn}, \gls{rf}, \gls{gb}, AdaBoost, \gls{dt}
and \gls{lda}. They evaluated their models on the CSE-CIC-IDS 2018 dataset. Due
to the class imbalance present in this dataset, the authors employ \gls{smote}.
The results indicate the proposed models are very effective with accuracies on
the sampled data ranging from 91.18\% to 99.35\%. The \gls{rf} algorithm had
the highest accuracy on the sampled dataset, however the AdaBoost algorithm had
the highest precision and F1 score which may be more significant given the
class imbalance present in the dataset.

Jiang et al. (2020)~\cite{Jiang} take a \gls{dl} approach to the problem. They
propose a \gls{cnn} to extract spatial features, which are then processed
through a \gls{bilstm} network to extract temporal features. To handle class
imbalance, the authors first apply \gls{oss} to reduce noisy samples in the
majority class, then increase the number of minority samples through
\gls{smote}. The proposed solution is evaluated on both the NSL-KDD dataset and
the UNSW-NB15 dataset, achieving accuracies of 83.58\% and 77.16\%
respectively.

Mighan and Kahani (2021)~\cite{Mighan} propose a hybrid approach using a
stacked auto-encoder network for latent feature extraction followed by a
support vector machine classifier. The proposed model was evaluated on the
ISCX2012 and CICIDS2017 datasets achieving accuracies of 90.2\% and 99.49\%
respectively. They concluded that \gls{dl} feature extraction outperformed
other methods. Note, class imbalance was not addressed in this study, despite
being present in both datasets used.

Kus et al. (2022)~\cite{Kus} argue that the current evaluation standard in
\gls{ml}-based industrial intrusion detection may create a false sense of
security. They argue that it does not assess the model's ability to detect
unknown attacks, absent from the training set. They propose a new evaluation
methodology to assess the efficacy of \gls{ml}-based industrial \gls{ids} in
detecting these unknown attacks. This methodology is applied to three
industrial intrusion detection models proposed in a work by Lopez Perez et
al.~\cite{Perez}. The results indicate an alarmingly low ability to recognise
unseen attacks, with detection rates dropping to between 3.2\% and 14.7\% for
some types of attacks. The authors conclude that the models they tested only
learn signatures of attacks and are not performing true anomaly-based intrusion
detection. The study focuses primarily on industrial intrusion detection and
does not attempt to make any generalisations to other domains of intrusion
detection.

A literature review carried out in 2023 by Ahmad et al.~\cite{zero-day} reviews
many more similar works with a particular focus on zero-day attacks. It
supports the argument of Kus et al.~\cite{Kus} stating that many researchers in
the field are not currently addressing the existence of zero-day attacks and
suggests that future works should not assume all existing attack
classifications are present in the dataset.

Pu et al.~\cite{Pu} propose an unsupervised approach, combining \gls{ssc} and
\gls{ocsvm}. They evaluate their proposal on the NSL-KDD dataset against three
other unsupervised models, specifically, a combination of \gls{ssc} and
Evidence Accumulation, \gls{dbscan} and K-means clustering. The results
indicated the proposed model outperformed the other models.

Cao et al.~\cite{Cao} take an unsupervised \gls{dl} approach, proposing new
regularisers to classical and variational auto-encoders to map normal network
samples towards the origin. One class classification models can then be used to
detect malicious samples which will be mapped further from the origin. They
evaluated this approach on 14 datasets, including, CTU13\-09, CTU13\-13,
NSL-KDD, and UNSW-NB15. The results indicate the approach is effective on high
dimensional and sparse network data.

In 2023, Zoppi et al.~\cite{Zoppi} carried out a comparison of 47 different
machine learning algorithms on 11 different network intrusion datasets. The
study includes supervised, unsupervised, \gls{dl} and meta-learning algorithms.
The methodology, like that of Kus et al.~\cite{Kus}, involves creating variants
of each dataset excluding specific categories in order to simulate the
occurrence of unknown attacks. They conclude that supervised classifiers yield
a higher accuracy, however, this is significantly degraded in the face of
unknown attacks. Unsupervised learners on the other hand are less accurate but
experience a less dramatic decline in the face of unknown attacks. They also
conclude meta-learners outperform their non-meta-learning counterparts.

Other, less conventional approaches have also been employed in the literature,
with some novel approaches having been proposed since this literature review
was first carried out. For instance, OpenMax is an extension of the softmax
activation function commonly used in \gls{dl} which has seen application to
\gls{nids}~\cite{openmax}. Another possibility that has been explored in a
different domain is \gls{gan} dataset enhancement which essentially involves
using \gls{dl} to generate realistic attack samples in the dataset without
prior knowledge of the attacks~\cite{gan_enhancement}. Biologically inspired
algorithms may also offer an interesting approach to the field~\cite{AIm}.

\section{Conclusion}%
\label{sec:conclusion2}
From the above works, we observe that the research community has achieved
impressive results in the field of intrusion detection. However, as Kus et
al.~\cite{Kus} have discovered, these laboratory results may not accurately
reflect reality.

The supervised learning approaches explored above tend not to consider the
possibility of unknown attacks and hence may be ineffective in practical
applications. As indicated by Zoppi et al.~\cite{Zoppi}, unsupervised
techniques may be more effective in detecting unknown attacks, however at the
cost of reduced efficacy against known attacks.
