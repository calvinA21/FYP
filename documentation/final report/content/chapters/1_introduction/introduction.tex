\chapter{Introduction}%
\label{chp:introduction}

% establish field

The ever-evolving landscape of cyberthreats poses a pervasive danger to all
facets of society, ranging from commercial enterprises to individuals and
government entities. As technology relentlessly advances and more aspects of
our lives become digitised, the risks posed by these threats continue to
escalate~\cite{ENISA}.

Cyberthreats refer to malicious acts carried out with the intention of
compromising the confidentiality, integrity, or availability of information
systems. Cyberthreats can have far-reaching consequences, ranging from the
exposure of private information to financial losses to the disruption of
critical systems. For instance, the 2015--2016 SWIFT banking hack resulted in
the theft of millions of dollars~\cite{swift_attack1, swift_attack2} from SWIFT
member banks. Another example is the ransomware attack on CommonSpirit Health,
which operates 140 hospitals and 2000 patient care sites. The attack led to the
disruption of health services and the theft of personal data, and in some cases
sensitive health data, of 623,774 patients~\cite{commonspirit1, commonspirit2,
    commonspirit3}. These examples clearly illustrate the importance of detecting
and preventing these cyberthreats.

There are numerous defence options available to help mitigate these threats,
prominent among them are \gls{nids}.\ \gls{nids} are a crucial part of any
information system's defence that aims to identify and mitigate network threats
in real time. Two major classifications exist that characterise the function of
intrusion detection systems~\cite{survey1}.\ \gls{sids} detect attacks by
matching sequences of code or commands with those of known attacks.\ \gls{aids}
on the other hand, detect threats by recognising deviations from normal,
non-malicious traffic~\cite{survey1}. These deviations will be referred to as
anomalies and typically represent intrusions, however may sometimes represent
false alarms~\cite{anomaly}.

Designing \gls{nids} suitable for the ever-evolving cyberthreat landscape is no
trivial task.\ \gls{nids} deployed practically are likely to face attacks not
known at the time of development, referred to as unknown attacks throughout
this study. These could take the form of new deviations of known attacks, new
attacks based on known vulnerabilities or zero-day attacks. Zero-day attacks
are cyberattacks executed on the basis of an exploit that is not known
publicly, and are particularly challenging to detect~\cite{zero-day-def}. Due
to their nature, \gls{sids} are ineffective in detecting unknown attacks, as by
definition, they cannot exist on any attack signature database~\cite{survey1}.

\gls{aids} exhibits promise through its ability to identify unknown
attacks~\cite{aids-unknown}, particularly with the advent of \gls{ml}
techniques.\ \gls{ml} techniques, typically considered a branch of
\gls{aids}~\cite{survey1}, offer enormous potential in anomaly detection as
they have the ability to model complex patterns which may not be known to the
developer in advance. These capabilities hold the promise of developing more
accurate and effective \gls{aids}.

In the literature, there is no absence of works investigating the use of
\gls{ml} in \gls{nids}~\cite{Karatas, Jiang, Mighan, Pu, Cao, Atefinia}. This
work explores various different approaches including supervised and
unsupervised techniques. The works of~\cite{Karatas, Jiang, Mighan, Atefinia}
explore supervised techniques and demonstrate remarkable accuracies as high as
100\%. These results offer great promise, with near perfect recall and very low
false alarm rate, indicating the model detects almost all attacks and seldom
flags benign traffic as an attack. Unsupervised techniques have also been
researched with results indicating a higher false alarm rate when compared to
supervised techniques~\cite{Zoppi}.\ \gls{ml} techniques can also be
categorised into traditional techniques and \gls{dl} techniques. The literature
explores both branches of \gls{ml}. Numerous studies focused on supervised
techniques have found traditional algorithms to be more effective in the field
of \gls{nids}~\cite{Liu, Zoppi}.

Some researchers have brought into question the notion that supervised
techniques can generalise to unknown attacks~\cite{Kus, Zoppi}. These studies
uncover a significant threat to many of the proposed models in the literature,
as the evaluation techniques used to measure the efficacy of the model may be
misleading. This could lead to a false sense of security that could put
information systems at serious risk to unauthorised access and misuse.

Ahmad et al.~\cite{zero-day} has noted that many studies focused on supervised
learning phrase their works as closed-set classification problems. Closed-set
classification problems have all possible classes available in the dataset,
whereas open-set classification problems may have other classes not known to
the dataset. Network intrusion detection is a dynamic field due to the constant
development of new and innovative attacks by threat actors, presenting an
open-set classification problem, calling further into question the efficacy of
current state-of-the-art models, on unknown attacks.

Unsupervised techniques do not rely on labelled data. It can therefore be
hypothesised that they are better suited for open set classification problems
as they can detect patterns without prior knowledge of specific
classes~\cite{unsupervised_ml}. This is more akin to true anomaly detection
whereby the algorithms focus on detecting deviations form benign traffic as
opposed to identifying specific attacks. This notion has been explored
previously in the literature by Zoppi et al.~\cite{Zoppi}, with results
supporting the hypothesis.

However, unsupervised techniques tend to have significantly higher false alarm
rates~\cite{Zoppi}, which hinders their efficacy in practical applications and
makes supervised approaches more appealing. Little research exists evaluating
the supervised approaches proposed in the literature in the context of an
open-set classification problem. Furthermore, the comparative studies found
during the literature review do not consider unsupervised \gls{dl}
approaches~\cite{Kus, Zoppi}, which may offer an alternative approach.

\section{Aims and Objectives}%
\label{sec:aims}

The principal aim of this study will be to investigate the efficacy of
supervised and unsupervised techniques proposed in the literature against both
known and unknown attacks, comparing the trade-offs between different
techniques.

This will take the form of three objectives:

\hypertarget{obj}{}

\begin{center}
    \fbox{\parbox{0.8\linewidth}{\textbf{Objective 1:} To determine the extent to which current state-of-the-art
            network intrusion detection models are able to detect both attacks that were present and absent in their
            training set.}}
\end{center}

This information is vital to measure the pragmatic performance of these models
as \gls{nids} deployed in practical circumstances will face a wide variety of
attacks, including unknown attacks.

% It should be noted that for the purposes of
% this study, we define state-of-the-art as any recently proposed model that
% demonstrates impressive efficacy and is highly cited, indicating widespread
% acceptance in the research community.

\begin{center}
    \fbox{\parbox{0.8\linewidth}{\textbf{Objective 2:} To investigate the extent to which unsupervised techniques are
            more or less effective against both known and unknown attacks compared to supervised
            techniques.}}
\end{center}

Unsupervised techniques do not rely on labelled attack data in their training.
Hence, they are theoretically better adapted to unknown attacks, however may be
less accurate on known attacks. This hypothesis is supported by the findings of
Zoppi et al.~\cite{Zoppi} and will be further explored in this study.

% \hypertarget{obj3}{}
\begin{center}
    \fbox{\parbox{0.8\linewidth}{\textbf{Objective 3:} To investigate the relationships between specific attacks and attack
            categories in regard to the generalisation ability of models trained on those attacks.}}
\end{center}

Certain attacks share similarities that may serve as the basis for models to
generalise to these attacks without explicit presence in the training set.
These similarities will be explored in this study.

To the best of our knowledge, this study is the first to include \gls{dl}
unsupervised techniques in a comparative analysis considering unknown attacks.

The remainder of this document is divided into four chapters. First, in
Chapter~\ref{chp:background}, we will explore the field of \gls{ml}-based
\gls{nids}, introduce important concepts in the field and discuss similar works
in the literature. Secondly, in Chapter~\ref{chp:methodology} we will explain
in detail the experiments proposed in this study to evaluate our objectives and
the implementation of these experiments. Next, in Chapter~\ref{chp:results}, we
will present the results of our experiments and discuss the interpretations and
significance of these results, as well as threats to their validity. Finally,
we will summarise and propose future potential research directions in the
field.