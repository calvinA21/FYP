\chapter{Methodology}%
\label{chp:methodology}

The work of Kus et al.~\cite{Kus} and Zoppi et al.~\cite{Zoppi} shines an
illuminating light onto the field. Hence, this study follows similar
principles.

Models are evaluated using the methodology of Kus et al.~\cite{Kus} as this is
a superset of the methodology applied by Zoppi et al.~\cite{Zoppi}.
Additionally, Kus et al.~\cite{Kus} consider individual attacks, which will
provide valuable insights to address \hyperlink{obj}{Objective 3}.

The methodology can be divided into the five following steps:
\begin{enumerate}
      \item Selection of state-of-the-art models.
      \item Selection of a realistic and diverse dataset.
      \item Sampling and preprocessing of the dataset.
      \item Generation of training variants which intentionally exclude certain attacks.
      \item Replication and evaluation of the replicated models on the training variants.
\end{enumerate}
The remainder of this chapter further details these steps.

\section{Selection of Models}%
\label{sec:models}

The first step to any comparative analysis is the selection of works for
comparison. To address \hyperlink{obj}{Objective 2}, we require at least one
supervised model and one unsupervised model. Additionally, we consider both
traditional and \gls{dl}-based unsupervised techniques. Hence, we aim to select
at least one model from three families. Namely, these families are supervised
traditional \gls{ml} techniques, unsupervised traditional \gls{ml} techniques
and unsupervised \gls{dl} techniques. It should be noted that supervised
\gls{dl} techniques are not considered as they have been found to be less
effective by Zoppi et al.~\cite{Zoppi}.

The criteria considered when selecting works were the number of citations by
other works in the literature, the results presented by the paper and the
specification of a clear replicable methodology. The former two criteria are
intended to ensure the works selected represent the state-of-the-art whilst the
latter criterion is intended to ensure accurate replication of the work is
possible.

Additionally, in the case of supervised techniques, works that consider class
imbalance were given preference as this has been identified as a significant
issue when applying supervised \gls{ml} to intrusion detection. This is due to
the prevalence of normal, non-malicious traffic in realistic scenarios when
compared to the volume of malicious traffic~\cite{imbalance_prob, imbalance}.

Based on these criteria, the works of Karatas et al.~\cite{Karatas}, Pu et
al.~\cite{Pu} and Cao et al.~\cite{Cao} were selected from the supervised
traditional \gls{ml}, unsupervised traditional \gls{ml} and unsupervised
\gls{dl} families respectively. Cao et al.~\cite{Cao} propose 14 models, only
one of which was selected for use in this study due to time constraints. The
\gls{sae}-\gls{ocsvm} model was selected as it is among the top performing
models evaluated by Cao et al.~\cite{Cao} and has a complete implementation
available in a Jupyter notebook in the repository provided by the authors.
Karatas et al.~\cite{Karatas} and Pu et al.~\cite{Pu} only propose seven models
in total, and hence, all of these are included in this study.

\section{Datasets}%
\label{sec:datasets}

Next, we select a dataset for use in our experiments. The aims of this study
are intended to better illustrate the performance of our selected models in a
realistic environment where unknown attacks may be encountered. Hence,
selecting a realistic, modern dataset with sufficient variety of data is a
vital step. In addition to the experimental dataset, we also require datasets
used by the original authors of the selected models. These are required so that
the replicated models can be evaluated using on this data using the same
metrics presented by the original authors. These metrics can then be compared
to those presented to verify the models were replicated correctly. Similarly,
the dataset used by Kus et al.~\cite{Kus} in the original proposal of their
methodology will also be used to evaluate the methodology and verify it has
been replicated correctly.

Our literature review yielded eight datasets that were considered for use in
this study. These include the KDD99~\cite{kdd99}, NSL-KDD~\cite{nsl},
CAIDA-DDOS2007~\cite{caida}, ISCX2012~\cite{iscx12}, UNSW-NB15~\cite{unsw15},
CIC-IDS2017~\cite{cic2017}, CSE-CIC-IDS2018~\cite{cic2018} and
IoT-botnet2020~\cite{iot_botnet20} datasets.

The CSE-CIC-IDS2018~\cite{cic2018} dataset is, to the best of our knowledge,
the most up-to-date, general purpose network intrusion dataset publicly
available today. It offers a large volume of data containing diverse attack
scenarios and a realistic network environment. It is for these reasons that it
has been identified as the most suitable dataset to evaluate and compare the
models of this study.

The CSE-CIC-IDS2018~\cite{cic2018} dataset is a collaborative effort between
the Communications Security Establishment and the Canadian Institute of
Cybersecurity, and serves as a successor to the CIC-IDS2017~\cite{cic2017}
dataset which is a smaller, less diverse counterpart. The aim of the project
was to produce an up to date, realistic dataset for the evaluation of
\gls{nids}. The dataset was generated on infrastructure consisting of a victim
network with 420 machines and 30 servers, and an attack network consisting of
50 machines. The machines were divided into departments mimicking a typical
corporate network. A typical network environment was simulated using
B-profiles, which describe user behaviour through statistical and \gls{ml}
techniques, allowing it to be accurately replicated. The dataset includes a
variety of modern attack categories, namely, \gls{dos} attacks, brute force
attacks, code injection, botnet attacks, and infiltration attacks which involve
using an exploit on a host inside the network to attack the network. The
resulting dataset contains 80 features and approximately 16,000,000
instances~\cite{cic2018, cic2018data}.

Karatas et al.~\cite{Karatas} evaluate their models on the
CSE-CIC-IDS2018~\cite{cic2018} dataset. Hence, this dataset can be used to
verify the replication of these models in addition to being used as the
experimental dataset. In contrast, the dataset is not considered by Pu et
al.~\cite{Pu} and Cao et al.~\cite{Cao}, therefore, another dataset is required
to verify the replication of these models. The work of Pu et al.~\cite{Pu} is
evaluated only on the NSL-KDD~\cite{nsl} dataset. This dataset is also one of
the datasets on which the models proposed by Cao et al.~\cite{Cao} are
evaluated. Hence, this dataset is used to verify the correct replication of
both these works.

The NSL-KDD~\cite{nsl} dataset is an improvement over its predecessor, the
KDD99~\cite{kdd99} dataset. It was created as a benchmark for intrusion
detection systems and offers a more balanced class distribution than its
predecessor. The dataset has been used extensively in the field of intrusion
detection, however, the underlying network data dates back to 1998. Due to its
age, the results may not generalise well to the modern cyber threat landscapes
and hence, results on this dataset cannot be considered when addressing the
aims of this study. However, this issue does not apply when comparing our
results with values generated from the same dataset, rendering this dataset
suitable for comparison with the work of the original authors of the
unsupervised models selected. This comparison allows us to verify our
replications are correct and accurate.

Finally, we require the Gas Pipeline dataset~\cite{gas_pipeline_dataset} used
by Kus et al.~\cite{Kus} in their original proposal of their methodology. The
dataset includes a variety of attacks carried out on a gas pipeline control
system in a lab environment. This will be used to verify the replication of the
methodology of Kus et al.~\cite{Kus}.

\section{Preprocessing}%
\label{sec:preprocessing}

The methodology of Kus et al.~\cite{Kus}, employed in this study, requires each
instance in the dataset to be labelled with both the specific attack name and
the attack category. The labels provided in the CSE-CIC-IDS2018~\cite{cic2018}
dataset can be described as an attack category, however, some labels are not
broad enough to be defined as categories for the purpose of this study. An
instance of this are the \gls{dos} attacks which are labelled according to the
specific tool used. Hence, the dataset has an `attack name' column appended to
it. The values of this column were filled according to Table 2 on the dataset's
website~\cite{cic2018}, using the `Timestamp' and `Label' columns to determine
the attack each instance belongs to.

Numerous instances in the dataset occur outside the specified attack times in
this table. In most cases, only one attack with a particular label occurred
each day. Hence, it has been assumed that instances with a matching label,
occurring outside the specified attack times, but on the same attack day
constitute part of that attack. The exception to this are the bot attacks
carried out on 02/03/2018. Two Bot attacks were carried out on the same day,
with malicious instances present between the end time of the first attack and
the start time of the second. These instances have been labelled as the
`Unknown' attack type.

Next, an `attack category' column was generated from this new column. The
categorisation adopted for this study is shown in Table~\ref{tab:categories}.
It should be noted that the previously mentioned ambiguous bot attacks were
categorised as `Bot' attacks.

\begin{table}
      \centering
      \caption{Attack Categorisation\label{tab:categories}}
      \begin{tblr}{|Q[m,0.15\textwidth]|Q[m,0.4\textwidth]|}
            \hline
            Category                     & Original Label           \\
            \hline
            Benign                       & Benign                   \\
            \hline
            \SetCell[r=4]{m} Brute Force & FTP-BruteForce           \\
                                         & FTP-BruteForce           \\
                                         & SSH-BruteForce           \\
                                         & Brute Force -Web         \\
            \hline
            \SetCell[r=7]{m} \gls{dos}   & DoS attacks-GoldenEye    \\
                                         & DoS attacks-Slowloris    \\
                                         & DoS attacks-SlowHTTPTest \\
                                         & DoS attacks-Hulk         \\
                                         & DDoS attacks-LOIC-HTTP   \\
                                         & DDOS attack-LOIC-UDP     \\
                                         & DDOS attack-HOIC         \\
            \hline
            Bot                          & Bot                      \\
            \hline
            Infiltration                 & Infilteration            \\
            \hline
            \SetCell[r=2]{m} Injection   & SQL Injection            \\
                                         & Brute Force -XSS         \\
            \hline
      \end{tblr}
\end{table}

As noted by Ahmad et al.~\cite{zero-day}, categorising attacks into broad
labels, such as \gls{dos}, can create inconsistencies and confusion in the
literature due to their subjectivity. Hence, the categorisations used in this
paper should be interpreted as a shorthand method of referring to the specific
attack types included under the category in this study. This shorthand serves
to simplify the process of investigating generalisation across attacks that
differ significantly in their nature and attacks that share similarities.
Conclusions are not be intended to generalise to attack types not present in
the dataset that may fall under the same subjective categorisation.

The CSE-CIC-IDS2018 dataset contains approximately 16,000,000 instances, an
enormous number that introduces complexities as it is not feasible to keep the
full dataset in memory during processing on the 32GB of RAM that were available
for this project. Hence, the dataset was sampled to 4,519,553 instances. The
sampling strategy adopted was random sampling of 28\% of the data.

Once the sampled, fully labelled dataset was generated, the individual
preprocessing pipelines of each replicated study were applied. Some additional
steps were added where necessary. We will now briefly outline the steps taken
in each work.

The following steps, based on the methodology of Karatas et al.~\cite{Karatas},
were carried out prior to all experiments.

\begin{enumerate}
      \item The CSE-CIC-IDS2018 dataset contains data points dated 1970 with illogical
            values in certain columns, likely caused by overflow errors during dataset
            generation. All instances of such data points were removed from the sample.
            Additionally, instances with an `Unknown' attack type were also removed.
      \item Missing values were replaced with zero
      \item Infinity values were replaced with the maximum value in the column
      \item During replication, the `Timestamp' columns was separated into date and time
            columns to eliminate non-numeric values. The column was then removed for the
            experiments as none of the models replicated analyse time series data. Hence,
            any information present in this column will be valid only under the specific
            conditions of the attacks present in the dataset, diminishing the realism of
            the experiments.
      \item The columns `Src IP', `Dst IP' and `Flow ID' similarly contain information
            specific to the attack configuration employed in the generation of the dataset.
            Hence, these have also been removed.
      \item Two columns contain negative values, `Init Fwd Win Byts' and `Init Bwd Win
            Byts'. Two additional categorical columns were added, containing a one for
            negative values and a zero for positive values in the corresponding column.
      \item The label column was encoded into numeric values. This was done for both the
            `attack category' and `attack name' columns for the purposes of this study.
      \item The dataset was split into training and testing sets using an 80/20 split and
            then shuffled.
      \item Lastly, \gls{smote} was employed on the training set to reduce the class
            imbalance present in the dataset. Minority classes were increased to at least
            5\% of the data when considering categories, and at least 1\% when considering
            individual attacks.
\end{enumerate}

Following these preprocessing steps, the unsupervised models had additional
preprocessing carried out, based on the work of the original authors. Note, the
\gls{smote} step was excluded when processing for unsupervised models as they
do not consider minority samples during training.

The following additional steps were applied to the data processed by the
\gls{ssc}-\gls{ocsvm} model, based on the methodology of Pu et al.~\cite{Cao}.
\begin{enumerate}
      \item The `Protocol' feature, which is the only categorical feature with more than
            two levels, was converted to one-hot encoding representation.
      \item F-test feature selection was employed, selecting the k best features based on
            the ANOVA F-value~\cite{f_test} of each feature, where k is a hyperparameter.
      \item All features were then standardised.
      \item Finally, malicious samples were removed from the training set, and the training
            set was reduced to 200,000 samples. This number was selected by progressivley
            increasing the sample size until the performance did not improve any further.
\end{enumerate}

The following additional steps were applied to the data processed by the
\gls{sae}-\gls{ocsvm} model, based on the methodology of Cao et al.~\cite{Cao}.
\begin{enumerate}
      \item The negative one values present in the columns `Init Fwd Win Byts' and `Init
            Bwd Win Byts' were replaced by -0.5 values. This is so that they scale back to
            their original negative one value following the log operation.
      \item All malicious samples were filtered out of the training set, and the training
            set was reduced to 6734 random samples. This number was selected as it was the
            number used by the original authors, and our experimentation with different
            sample sizes failed to yield better results.
      \item Next, columns containing large values (larger than 10,000) had one added to
            them followed by the logarithm operation base two.
      \item Finally, each feature was scaled by its maximum absolute value.
\end{enumerate}

\section{Training Variants}%
\label{sec:variants}

Following preprocessing, the dataset was split into variants according to the
methodology of Kus et al.~\cite{Kus}. Briefly, this consists of filtering the
dataset in four phases. In the first phase, the dataset has particular attack
excluded from the training set. Hence, the excluded attacks simulate unknown
attacks and accurately depict model performance on such unknown attacks should
it encounter them at runtime. In the second phase, all attacks are filtered out
except for one attack, and of course, the benign class. This provides more
insight allowing for a better understanding of how individual attacks provide
valuable knowledge that may generalise to other attacks. In the third and
fourth phases, these two steps are repeated using attack categories instead of
particular attack types. This provides more insight as to how attacks may
generalise to other attacks within the same categories as well as attacks in
other categories.

The CSE-CIC-IDS2018 contains 21 individual attacks, which have been divided
into five categories. Hence, 52 dataset variants were generated.

\section{Implementation and Execution}%
\label{sec:implementation}

The evaluation of this study is written in Python 3.11. The \gls{dt}, \gls{rf},
\gls{knn} and \gls{lda} models were implemented using the scikit-learn
library~\cite{scikit-learn} as was done by the original authors. The \gls{gb}
model was implemented using the XGBoost library~\cite{xgboost} to increase the
execution speed beyond what is possible with scikit-learn~\cite{scikit-learn}.
The work of Pu et al.~\cite{Pu} was replicated using
ThunderSVM~\cite{ThunderSVM}, a \gls{gpu}-accelerated library that offers an
\gls{ocsvm} implementation. Cao et al.~\cite{Cao} provide the source code of
their implementation on GitHub~\cite{cao_git}. This implementation employs
Tensorflow~\cite{tensorflow} for the \gls{sae} and scikit-learn for the
\gls{ocsvm} model. The \gls{sae} implementation was updated to work with
Tensorflow 2 and the \gls{ocsvm} model was replaced with a
ThunderSVM~\cite{ThunderSVM} model to reduce execution time.

Experiments are executed on \gls{wsl} Ubuntu 22.04 with an Intel I9--10900F,
two \gls{gpu} NVIDIA GeForce RTX3080 with 10 GB \gls{ram} each and 32 GB main
\gls{ram}\@ The ThunderSVM~\cite{ThunderSVM} and Tensorflow~\cite{tensorflow}
libraries were configured to leverage \gls{gpu} acceleration.

The results of each original study were replicated on one of the datasets used
by the original authors to verify the models were replicated correctly. The
methodology was also replicated on the models and dataset used by Kus et
al.~\cite{Kus} to ensure it was replicated correctly. Once, matching results
were achieved, the models were trained on the variants to produce the results
of this study. The unsupervised models considered expect only normal, or
benign, data in their training sets. Hence, these were only trained once as all
variants are rendered identical when malicious samples are omitted. Each model
was trained as a binary classification model, classifying each sample as benign
or malicious.

Metrics were then calculated based on the predictions obtained during
execution. The scikit-learn~\cite{scikit-learn} classification report and
accuracy score functions were used to compute the metrics. To better define
performance on unknown attacks, we consider the Recall-Unk metric defined by
Zoppi et al.~\cite{Zoppi}. This metric represents the recall value when
considering only samples of the unknown class and provides insight into the
performance of the model on unknown classes.

% In the ideal case, the evaluation metrics of 
% each dataset variant would be calculated using 5-fold cross validation.
% However, due to the high computational demands of the methodology, there is a
% possibility that the proposed procedure will be unfeasible to carry out using
% the resources available. Should this be the case, a standard train-test split
% may be employed in place of 5-fold cross validation and stratified sampling can
% be explored to reduce the size of the dataset without decreasing the quality of
% the results.

The hyperparameters used are specified in Table~\ref{tab:hyperparameters}. All
source code used in the implementation can be found at
https://github.com/calvinA21/FYP/.

\begin{table}
      \caption{Hyperparamters\label{tab:hyperparameters}}
      \centering
      \begin{tblr}{|Q[m,0.15\textwidth]|Q[m,0.8\textwidth]|}
            \hline
            \textbf{Model}        & \textbf{Hyperparameters}                                                                  \\
            \hline
            \gls{dt}              & splitter=`best' criterion=`Gini' min\_samples\_split=2 min\_samples\_leaf=1               \\
            \gls{rf}              & n\_estimators=100 min\_samples\_split=2 min\_samples\_leaf=1 criterion=`Gini              \\
            \gls{gb}              & loss=`log\_loss' learning\_rate=1 n\_estimators=100 max\_depth=3 validation\_fraction=0.1 \\
            \gls{knn}             & n\_neighbours=5 weights=`uniform' metric=`Minkowski'                                      \\
            \gls{lda}             & solver=`svd'                                                                              \\
            \gls{ssc}-\gls{ocsvm} & k\_features=50 nu=0.1                                                                     \\
            \gls{sae}-\gls{ocsvm} & lambda=10 learning\_rate=0.01 n\_neurons=(95,49,12) nu=0.5 kernel=`rbf' gamma=0.1         \\
            \hline
      \end{tblr}
\end{table}
\section{Conclusion}%
\label{sec:conclusion3}

In this chapter we have presented the selected state-of-the art models as well
as our criteria during the selection process. We then outlined the data used in
our replications and experiments as well as our methodology for creating
variants of this data to simulate unknown attacks. Finally, we present our
preprocessing pipeline and implementation details for all models considered. In
the next chapter, we will present the results from all our replications and
experiments and discuss the significance of these results.