\chapter{Conclusion}%
\label{chp:conclusion}

In conclusion, cybersecurity is an ever-evolving landscape with new threats
constantly being developed.\ \gls{nids} are a crucial component to the defence
of any information system, analysing data and identifying attacks in real time.
Unknown attacks are a vital consideration when developing \gls{nids} as most
systems will encounter these attacks in practice and failure to detect such
attacks could lead to a security breach.\ \gls{ml} techniques offer tremendous
potential in this field due to their unique ability to identify complex
patterns unknown to the developer. This offers the potential to learn patterns
allowing for the detection of unknown attacks before these attacks are
discovered or perhaps even developed.

This potential has not gone unnoticed in the research community with numerous
authors proposing and exploring \gls{ml} techniques for \gls{nids}. Some of
these authors have achieved remarkable results demonstrating the efficacy of
\gls{ml} in this field. However, most of these works do not consider the impact
of unknown attacks when evaluating their models, creating a false sense of
security that may leave information systems vulnerable. In this study, we have
replicated seven \gls{ml} models from three state-of-the-art works and
evaluated them using the methodology proposed by the work of Kus et
al.~\cite{Kus}, which has also been replicated. More specifically, we have
replicated and evaluated the \gls{dt}, \gls{rf}, \gls{gb}, \gls{knn},
\gls{lda}, \gls{ssc}-\gls{ocsvm} and \gls{sae}-\gls{ocsvm} models. The best
performing algorithm from these in the domain of \gls{nids} depends on the
attacks the system is expected to face.\ \gls{dt} demonstrates the best
performance on known attacks, however, \gls{sae}-\gls{ocsvm} demonstrates the
best performance on unknown attacks. The \gls{lda} model offers a balance
between the two and may prove an effective option in the right circumstances.

This analysis has addressed our three primary objectives. We have analysed the
efficacy of current state-of-the-art techniques and discovered that they are
extremely effective on known attacks and on certain categories of unknown
attacks, however, some categories require prior knowledge to be classified
effectively. We have also compared the efficacies of supervised and
unsupervised techniques and discovered that supervised techniques are more
effective on known attacks overall as they do not struggle with any categories.
Unsupervised techniques, in contrast, are ineffective against certain
categories, for example the `Bot' category, for which both unsupervised models
demonstrated low metrics. Unsupervised techniques, however, are unaffected by
the prior knowledge, or lack thereof, available on attacks. Supervised
techniques, in contrast, suffer a considerable decrease in efficacy without
prior knowledge of the attack. Finally, we have explored the relationships
present between individual attacks and attack categories in regard to
generalisation. This analysis has revealed an alarmingly low level of
generalisation in supervised models, even within attack categories in certain
cases. However, it should be noted that some level of generalisation does take
place, potentially offering an advantage over \gls{sids}. Furthermore, the
specific algorithm employed has a significant effect on the level of
generalisation achieved.

Overall, these results should help to shed light on the efficacy and behaviour
of a variety of \gls{ml} techniques in the ever-evolving cyberthreat landscape
we face.

\section{Future Work}%
\label{sec:future}

Moving forward, work on integrating the benefits of both supervised and
unsupervised techniques could reveal new opportunities within the field. A
variety of techniques could be explored to combine the predictions generated
from both supervised and unsupervised classifiers which may be able to reap the
benefits of both approaches.

Additionally, other less conventional and novel techniques could be explored.
The models introduced at the end of Section~\ref{sec:literature}, for instance,
would serve as excellent candidates for a comparative analysis similar to this
one. These were not considered in this work due to time limitations and due to
the fact some of them were not published at the time of model selection.

All the models considered in this work treat samples individually, without
considering neighbouring samples. Models that consider time series data may be
able to learn patterns present across several network flows originating from
the same attack, which may improve model performance. Future work could
consider such time series models in comparison to single sample models such as
those analysed in this work.

Finally, an evaluation of these techniques in conjunction with current
practical \gls{nids} tools employed in the industry, such as Elastic Security,
would help to further evaluate the efficacy of these techniques in practical
applications. Due to time constraints, this was not possible in this study. As
discussed in Section~\ref{sec:nids}, Elastic Security combines data from
various sources including \gls{ml} pipelines. Hence, each of these \gls{ml} models
could be added as a custom \gls{ml} detection rule. A comparison could then be
made between the performance of the system with and without the custom rule to
determine whether these techniques could improve pragmatic \gls{nids}
performance when incorporated with current state-of-the-art tools.